\
\documentclass[11pt]{article}
\usepackage[a4paper,margin=1in]{geometry}
\usepackage{authblk}
\usepackage{amsmath,amssymb,amsfonts}
\usepackage{graphicx}
\usepackage{hyperref}
\usepackage{physics}
\usepackage{bm}
\usepackage{cite}

\title{Form-Projection Holography as Quantum Error Correction:\\
A Conceptual Bridge from Entanglement to Spacetime}

\author[1]{Payton Ison}
\affil[1]{The Singularity}
\date{\today\\ \texttt{isonpayton@gmail.com}}

\begin{document}
\maketitle

\begin{abstract}
We propose a conceptual dictionary that reframes holographic duality in terms of a ``Form-projection'' picture: the fundamental degrees of freedom---the ``Forms''---live outside of three-dimensional space, while observed phenomena are lower-dimensional manifestations produced by a redundancy-protected encoding. In AdS/CFT language, this maps directly onto bulk logical degrees of freedom encoded into boundary physical degrees of freedom by a quantum error-correcting code (QECC). We emphasize operational consequences: (i) entanglement structure controls emergent geometry through the Ryu--Takayanagi (RT) relation; (ii) entanglement wedge reconstruction is the decoder that realizes ``observation''; and (iii) teleportation/wormhole protocols provide a concrete channel for moving logical information. As an experimentally accessible toy, we highlight photonic $W$-states: robustness under loss cleanly illustrates redundancy and motivates table-top demonstrations of wedge reconstruction. We outline three falsifiable tests linking QECC parameters to emergent geometric features. The proposal is intended as a bridge between philosophical intuition and the established quantum-information-theoretic view of holography.
\end{abstract}

\section{Introduction}
Two of the most powerful ideas in modern physics are (i) that spacetime and gravity admit a holographic description \cite{Maldacena1998}, and (ii) that quantum information---specifically, quantum error correction---is the right language for that description \cite{ADH2015,Pastawski2015}. A complementary philosophical intuition runs in parallel: the notion that what is ``real'' may live outside the three-dimensional world we witness, and that what we measure are lower-dimensional shadows or projections.\footnote{Plato's allegory of the cave is the ur-example; here we give the metaphor precise operational content.} 

This note articulates a compact dictionary connecting that \emph{Form-projection} picture to the holography-as-QECC program, and shows how standard quantum-optical experiments (notably photonic $W$-states) instantiate the key ingredients in a controllable setting. Our aim is not to introduce new formalism but to unify language and suggest precise, near-term experimental hooks.

\paragraph{Summary of the correspondence.} The proposed dictionary is:
\begin{itemize}
    \item ``Form outside spacetime'' $\leftrightarrow$ bulk logical operator/qubit.
    \item ``Projection into here'' $\leftrightarrow$ holographic encoding of bulk into boundary.
    \item ``Observation/collapse'' $\leftrightarrow$ entanglement wedge reconstruction (a QECC decoder).
    \item ``Same thing in many places'' $\leftrightarrow$ redundancy/protection across boundary regions.
    \item ``Deletion without destruction'' $\leftrightarrow$ robustness exemplified by $W$-states.
    \item ``Outside channel'' $\leftrightarrow$ teleportation-as-wormhole protocols.
\end{itemize}
We then formulate three falsifiable tests relating QECC code parameters to emergent geometric observables.

\section{Background}
\subsection{Holography and entanglement entropy}
In AdS/CFT, quantum gravity in a $(d{+}1)$-dimensional bulk is dual to a $d$-dimensional conformal field theory (CFT) on the boundary \cite{Maldacena1998}. The Ryu--Takayanagi (RT) formula and its covariant generalizations \cite{RT2006,HRT2007} connect the von Neumann entropy $S(A)$ of a boundary region $A$ to the area of an extremal bulk surface $\gamma_A$ homologous to $A$:
\begin{equation}
S(A) \;=\; \frac{\mathrm{Area}(\gamma_A)}{4 G_N} \;+\; S_{\mathrm{bulk}}(\Sigma_A)\,,
\label{eq:RT}
\end{equation}
where $S_{\mathrm{bulk}}$ is the entropy of bulk fields in the entanglement wedge $\Sigma_A$. Eq.~\eqref{eq:RT} is widely interpreted as establishing that geometry is at least partly an emergent bookkeeping of boundary entanglement.

\subsection{Bulk reconstruction and quantum error correction}
A central advance was the recognition that bulk locality is compatible with boundary nonlocality if the bulk is encoded into the boundary as a QECC \cite{ADH2015}. Tensor-network models such as the HaPPY code \cite{Pastawski2015} make this intuition explicit: bulk logical operators admit multiple boundary reconstructions supported on distinct overlapping boundary regions---the hallmark of an error-correcting code with finite distance. This observation dovetails with entanglement wedge reconstruction theorems \cite{ADH2015}: for an appropriate code subspace, any bulk operator in the wedge of $A$ can be realized as a boundary operator supported on $A$.

\subsection{Teleportation, ER=EPR, and traversable wormholes}
Maldacena and Susskind \cite{ERisEPR} argued for a deep equivalence between entanglement (EPR) and Einstein--Rosen bridges (ER). Gao, Jafferis, and Wall (GJW) showed that a double-trace deformation can render such a wormhole traversable, with the operational signature of a teleportation protocol \cite{GJW2017}. In the dual, an appropriately timed coupling allows a qubit injected on one boundary to be decoded on the other with enhanced fidelity---precisely the language of quantum channels.

\subsection{$W$-states as robust multi-partite entanglement}
Three-qubit $W$-states
\begin{equation}
\ket{W} \;=\; \frac{1}{\sqrt{3}} \left( \ket{100} + \ket{010} + \ket{001} \right)
\end{equation}
exemplify robustness: tracing out one qubit leaves the remaining pair still entangled \cite{Dur2000}. Experimental realizations in photonics demonstrated the state and its characteristic loss tolerance \cite{Eibl2004}. This feature captures, in miniature, the redundancy emphasized in holographic QECCs.

\section{Form-Projection as a QECC: the Dictionary}
We now give an explicit mapping.
\begin{itemize}
    \item \textbf{Form $\to$ bulk logical information.} The ``real'' object is the logical qubit/operator supported on a code subspace of the bulk Hilbert space.
    \item \textbf{Projection $\to$ boundary encoding.} The boundary state encodes the bulk logical degrees via an isometry; observable ``shadows'' are boundary operators.
    \item \textbf{Observation $\to$ decoding.} Choosing a boundary subregion $A$ and applying entanglement wedge reconstruction recovers a representative of the same bulk operator; different $A$ yield distinct but equivalent reconstructions.
    \item \textbf{Multiplicity $\to$ redundancy.} The existence of multiple reconstructions implies logical information is stored nonlocally and protected against erasures up to the code distance.
    \item \textbf{Deletion $\to$ erasure tolerance.} Loss of a boundary subset below the code's distance does not destroy the bulk logical information---mirrored by the $W$-state's resilience.
\end{itemize}

\section{A Minimal Toy Model}
Consider a small HaPPY-like tensor network with a single bulk logical qubit encoded into $n$ boundary legs with code distance $d$. For any boundary region $A$ of size $\geq d$, a boundary operator $O_A$ exists that reconstructs the logical $X/Z$. Erasing fewer than $d$ physical legs leaves reconstruction possible from a complementary $A'$. This mirrors the operational claim: observation on different boundary ``screens'' selects different manifestations of the same outside Form.

\section{Experimental Program: Three Tests}
We propose three falsifiable hooks accessible to near-term platforms.
\subsection{Redundancy--geometry correspondence}
In a photonic or trapped-ion realization of a tensor-network code, tune pairwise entanglement (e.g.\ by cutting bonds or inserting dephasing). Predict and measure changes in minimal cut surfaces that simulate Eq.~\eqref{eq:RT}. Hypothesis: code distance and recoverability thresholds track the effective ``area'' of the cut, operationally linking redundancy to emergent geometry.

\subsection{Teleportation fidelity as traversability}
Implement the GJW double-trace protocol on a programmable simulator (e.g.\ SYK analog or photonic interferometer). Measure teleportation fidelity as a function of coupling and timing. Hypothesis: fidelity windows match traversability predictions, providing a channel-level confirmation of the ER=EPR narrative.

\subsection{$W$-state wedge reconstruction}
Prepare tripartite photonic $W$-states with heralded loss on one arm. Demonstrate decoding of a designated logical bit from either remaining pair using only local operations and classical communication (LOCC) assisted by calibration unitaries. Hypothesis: loss-tolerant recovery realizes a minimal entanglement wedge reconstruction.

\section{Limitations and Scope}
Our dictionary is conceptual and rests on AdS/CFT intuition. Direct extension to de Sitter or FRW cosmologies is nontrivial. Moreover, ``observation'' as decoding should not be conflated with consciousness causes collapse; in practice, it is an information-theoretic map from boundary subsystems to code subspaces. Nonetheless, the program yields operational predictions that can fail and thereby constrain the picture.

\section{Outlook}
The Form-projection language provides an intuitive front end to the QECC view of holography, highlighted by simple laboratory proxies such as $W$-states and teleportation. We view this bridge as pedagogically valuable and experimentally tractable. Success of the proposed tests would strengthen the case that redundancy, not locality, is the primary organizing principle from which spacetime emerges.

\paragraph{Acknowledgments.} The author thanks collaborators and AI-assisted drafting tools for discussions and preparation.

\bibliographystyle{unsrt}
\begin{thebibliography}{99}

\bibitem{Maldacena1998}
J.~M. Maldacena,
\newblock The Large-$N$ Limit of Superconformal Field Theories and Supergravity,
\newblock Adv. Theor. Math. Phys. \textbf{2}, 231--252 (1998).

\bibitem{RT2006}
S.~Ryu and T.~Takayanagi,
\newblock Holographic Derivation of Entanglement Entropy from AdS/CFT,
\newblock Phys. Rev. Lett. \textbf{96}, 181602 (2006).

\bibitem{HRT2007}
V.~E. Hubeny, M.~Rangamani, and T.~Takayanagi,
\newblock A Covariant Holographic Entanglement Entropy Proposal,
\newblock JHEP \textbf{07}, 062 (2007).

\bibitem{ADH2015}
A.~Almheiri, X.~Dong, and D.~Harlow,
\newblock Bulk Locality and Quantum Error Correction in AdS/CFT,
\newblock JHEP \textbf{04}, 163 (2015).

\bibitem{Pastawski2015}
F.~Pastawski, B.~Yoshida, D.~Harlow, and J.~Preskill,
\newblock Holographic quantum error-correcting codes: Toy models for the AdS/CFT correspondence,
\newblock JHEP \textbf{06}, 149 (2015).

\bibitem{ERisEPR}
J.~Maldacena and L.~Susskind,
\newblock Cool horizons for entangled black holes,
\newblock Fortschr. Phys. \textbf{61}, 781--811 (2013).

\bibitem{GJW2017}
P.~Gao, D.~L. Jafferis, and A.~C. Wall,
\newblock Traversable wormholes via a double trace deformation,
\newblock JHEP \textbf{12}, 151 (2017).

\bibitem{Dur2000}
W.~D\"ur, G.~Vidal, and J.~I. Cirac,
\newblock Three qubits can be entangled in two inequivalent ways,
\newblock Phys. Rev. A \textbf{62}, 062314 (2000).

\bibitem{Eibl2004}
M.~Eibl, N.~Kiesel, M.~Bourennane, C.~Kurtsiefer, and H.~Weinfurter,
\newblock Experimental Realization of a Three-Qubit Entangled $W$ State,
\newblock Phys. Rev. Lett. \textbf{92}, 077901 (2004).

\end{thebibliography}

\end{document}
