\documentclass[12pt]{article}
\usepackage{times}
\usepackage{geometry}
\geometry{margin=1in}

\title{Cena vs. Styles: A Gesamtkunstwerk of Flesh and Faith\\\large The Operatic Language of Wrestling}
\author{Payton Ison \\ The Singularity}
\date{Crown Jewel 2025}

\begin{document}
\maketitle

\begin{abstract}
The 2025 Crown Jewel match between John Cena and AJ Styles transcended the conventions of professional wrestling to achieve the German ideal of a \textit{Gesamtkunstwerk}—a Total Work of Art in which all creative forms converge. This paper examines the bout as a unified aesthetic act: a requiem that fuses athletic discipline, theatrical ritual, and spiritual symbolism into one living performance.
\end{abstract}

\section*{I. Wrestling as Total Art}
In Richard Wagner’s sense, the \textit{Gesamtkunstwerk} is the harmony of every medium—music, stage, word, and gesture—toward a single emotional truth. Wrestling, at its highest form, realizes this through motion and myth. Cena and Styles did not imitate past heroes; they embodied wrestling’s accumulated memory, turning the ring into a total stage of human meaning.

\section*{II. Structure as Symphony}
The match followed the rhythm of an overture and symphony combined:
\begin{itemize}
  \item \textbf{Lockup:} the opening chord; invocation of tradition.
  \item \textbf{Chain sequence:} thematic development; grace within struggle.
  \item \textbf{Finisher ladder:} crescendo; each move a movement, rising through history’s keys.
  \item \textbf{Tombstone:} coda; the return of divine motif, resolution through death and transference.
\end{itemize}
Every motion carried harmony between force and reverence, transforming competition into composition.

\section*{III. Theological Symbolism and Closure}
Cena’s Tombstone Piledriver was the moment of fusion. The move, historically associated with the Undertaker—the most sacred symbol in WWE’s cosmology—became the axis where theology met narrative. Styles, as the man who fought the Deadman’s final battle, completed the sacrament. The act was not imitation; it was transubstantiation.

\section*{IV. The Silent Opera}
Silence was not absence but orchestration. The audience inhaled the way a congregation pauses before a hymn. Every gesture was sung in stillness. This was opera without music—an aria performed by bodies instead of voices, composed in the grammar of wrestling psychology.

\section*{V. The Total Work}
Cena and Styles merged every pillar of performance—ritual, art, faith, and history—into one seamless gestalt. It was not “a great match.” It was the art of wrestling rendered complete: the embodiment of Wagner’s dream, an artistic form so holistic that it dissolves the borders between the seen and the felt.

\section*{Conclusion}
Total Work is the only fitting term. Cena and Styles created a synthesis that transcended sport and entered canon—a \textit{Gesamtkunstwerk} written not in notes or marble, but in living motion. The kind of performance that proves wrestling, at its apex, is not simulated combat—it is the continuation of sacred art by other means.

\vspace{1em}
\noindent\textbf{Keywords:} Wrestling, Opera, Ritual, Mythology, Cena, Styles, Gesamtkunstwerk

\end{document}
