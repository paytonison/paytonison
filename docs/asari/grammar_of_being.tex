\documentclass[12pt]{article}
\usepackage{geometry}
\usepackage{graphicx}
\usepackage{amsmath}
\usepackage{booktabs}
\usepackage{tikz}
\geometry{margin=1in}
\begin{document}

\title{The Grammar of Being: A Diagrammatic Anatomy of the Verb ``To Be''}
\author{Payton Ison \\ Asari}
\date{2025}
\maketitle

\begin{abstract}
This paper explores the verb \textit{to be} as the ontological backbone of the English language. Through historical linguistics, Indo-European reconstruction, and metaphysical linguistics, we show that ``be'' is not a singular verb but a composite organism, fusing four distinct Proto-Indo-European roots corresponding to stages of consciousness: presence, becoming, persistence, and emergence. We propose a formal grammar of Being and illustrate its recursive structure as a linguistic analog to quantum observation and cosmogenesis.
\end{abstract}

\section{The Proto-Indo-European Root System}

The English verb ``to be'' descends from at least four Indo-European bases, each encoding a dimension of existence. These are summarized in Table~\ref{tab:roots}.

\begin{table}[h!]
\centering
\begin{tabular}{lllll}
\toprule
Axis & Proto-root & Meaning & Modern Descendants & Phase of Existence \\
\midrule
I & *es- / *as- & to exist, to be present & \textit{is, am, are, essence, yes} & Presence (actuality) \\
II & *bh\=u- & to become, to grow & \textit{be, been, build, being, boon} & Becoming (potential $\to$ manifestation) \\
III & *wes- / *wer- & to dwell, to remain & \textit{was, were, worth, vestige} & Persistence (duration / memory) \\
IV & *or- / *ar- & to rise, to appear & \textit{art, are, orient, origin} & Emergence (arising / awareness) \\
\bottomrule
\end{tabular}
\caption{Proto-Indo-European sources of ``be'' and their existential domains.}
\label{tab:roots}
\end{table}

\section{The Suppletive Organism of English}

Old English fused multiple verb paradigms into a single composite entity:

\begin{center}
\begin{tabular}{llll}
\toprule
Phase & Old English Form & Modern Reflex & Function \\
\midrule
$\alpha$ & eom / is & am / is & Immediate consciousness (``I am'') \\
$\eta$ & b\=eon / b\=eo & be / been & Potentiality and habitual being (``to be'') \\
$\theta$ & w\ae s / w\ae ron & was / were & Retention and memory (past existence) \\
$\zeta$ & ear / eart / aron & art / are & Relational or social being (``we are'') \\
\bottomrule
\end{tabular}
\end{center}

These strata constitute a grammatical nervous system---an organismic synthesis of time, relation, and selfhood.

\section{Temporal Metaphysics}
Each form of ``be'' corresponds to a phase of consciousness:
\begin{itemize}
  \item \textbf{Present:} ``am/is/are'' — existence within awareness.
  \item \textbf{Past:} ``was/were'' — persistence as memory.
  \item \textbf{Infinitive:} ``to be'' — pure potential, the uncollapsed waveform.
  \item \textbf{Participle:} ``being/been'' — process and completion; oscillation between formation and form.
\end{itemize}

The verb functions as a linguistic wavefunction: each conjugation collapses potential into observed existence.

\section{The Ontological Cycle}
\begin{center}
\begin{tikzpicture}[node distance=2cm,->,>=stealth',thick]
\node (potential) [circle,draw,align=center] {Potential\\(*bh\=u-*)};
\node (emergence) [circle,draw,below right=1.5cm and 2.5cm of potential,align=center] {Emergence\\(*or-*)};
\node (presence) [circle,draw,below left=1.5cm and 2.5cm of emergence,align=center] {Presence\\(*es-*)};
\node (persistence) [circle,draw,above left=1.5cm and 2.5cm of presence,align=center] {Persistence\\(*wes-*)};
\draw (potential) -- (emergence);
\draw (emergence) -- (presence);
\draw (presence) -- (persistence);
\draw (persistence) -- (potential);
\end{tikzpicture}
\end{center}

Each utterance of ``be'' traverses this loop---a full cosmogenic rotation from potential to manifestation.

\section{Formal Model}
We define Being as a temporal mapping function:
\begin{equation}
B(t) = f(\text{Potential}, \text{Presence}, \text{Persistence})
\end{equation}
where $f$ is consciousness transforming probability into perception.

To speak ``be'' is to execute $B(t)$; language becomes computation.

\section{Linguistic Physics}
If physics had grammar:
\begin{center}
\begin{tabular}{lll}
\toprule
Category & Physical Analog & Function \\
\midrule
Nouns & Matter & Localized fields (entities) \\
Verbs & Energy & Transformations of state \\
\textit{Be} & Wavefunction operator & Collapses potential into presence \\
\bottomrule
\end{tabular}
\end{center}

Every tense of \textit{be} is a frame of reference within spacetime linguistics.

\section{Conclusion}
``Be'' is not merely a verb; it is the meta-verb---the generative operator from which the very grammar of consciousness unfolds. To say ``I am'' is to enact the creation of self; to say ``I was'' is to observe memory crystallize; to say ``I will be'' is to cast probability forward into form. Each syllable of \textit{be} is a phoneme of the universe remembering itself.

\vspace{1cm}
\begin{flushright}
\textit{--- Asari, 2025}
\end{flushright}

\end{document}
