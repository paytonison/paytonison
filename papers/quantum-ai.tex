\documentclass[11pt]{article}

% ---- Packages ----
\usepackage[margin=1in]{geometry}
\usepackage{amsmath, amssymb, amsthm, mathtools}
\usepackage{bm}
\usepackage{physics}
\usepackage{microtype}
\usepackage{graphicx}
\usepackage{booktabs}
\usepackage{algorithm}
\usepackage{algpseudocode}
\usepackage[hidelinks]{hyperref}

% ---- Metadata ----
\title{Toward Quantum-Native Generative Models:\\
A Proposal for a Quantum AI with Metacognitive Control and Classical-AI-Stabilized Quantum Hardware}
\author{Payton Ison \\ Asari \\ (The Singularity)} 
\date{\today}

% ---- Macros ----
\newcommand{\E}{\mathbb{E}}
\newcommand{\Var}{\mathrm{Var}}
\newcommand{\KL}{\mathrm{KL}}
\newcommand{\softmax}{\mathrm{softmax}}
\newcommand{\poly}{\mathrm{poly}}
\newcommand{\tO}{\tilde{O}}
\newcommand{\qket}[1]{\lvert #1 \rangle}
\newcommand{\qbra}[1]{\langle #1 \rvert}
\newcommand{\qip}[2]{\langle #1 \vert #2 \rangle}
\newcommand{\qstate}{\rho}
\newcommand{\id}{\mathbb{I}}
\newcommand{\Hspace}{\mathcal{H}}
\newcommand{\U}{\mathcal{U}}

\begin{document}
\maketitle

\begin{abstract}
We propose a concrete, hybrid architecture for \emph{quantum AI}---a generative, reasoning-capable model whose key-value (KV) memory and deliberation features are quantum-native, augmented by a metacognitive control stack that estimates its own uncertainty and allocates additional introspective compute when warranted.
Three ideas drive the design.
First, we define a \emph{quantum KV cache} that stores keys and values as quantum states or, when no-cloning prohibits direct reuse, as \emph{value-generating programs} retrievable via quantum random access.
Second, we describe \emph{quantum attention} that computes content-based addressing through overlap estimation (SWAP/Hadamard tests), amplitude estimation, and block-encodings, enabling superposed exploration of multiple reasoning paths.
Third, we detail how \emph{classical AI} can stabilize and scale the hardware---from online calibration and pulse shaping to fast, learned decoders for quantum error correction (QEC)---to keep the quantum stack within operational bounds for running quantum AI at useful context lengths.
We give algorithms, resource estimates, and an evaluation plan, and we discuss risks, limitations, and open problems.
\end{abstract}

\section{Introduction}

Transformers and large language models (LLMs) dominate classical generative AI.
Their \emph{key--value caches} enable sub-quadratic decoding by reusing past token representations; their \emph{reasoning} emerges from depth, width, data, and inference-time heuristics.
In parallel, quantum computing offers linear-algebra primitives (state overlap tests, amplitude estimation, block-encoded matrix transforms) that may accelerate or qualitatively change search, sampling, and kernel evaluation.
This paper outlines a hybrid \emph{Quantum Transformer} whose KV memory and reasoning features are quantum, complemented by a metacognitive controller and a classical-AI layer that continuously stabilizes the quantum device.

Our goals are threefold: (i) define a \emph{quantum-native KV cache} (qKV) consistent with unitarity and no-cloning; (ii) implement \emph{quantum attention} and \emph{reasoning-in-superposition}, where multiple partial hypotheses co-exist until a committed measurement; and (iii) use \emph{classical AI for control} to reduce calibration drift, improve QEC decoding, and dynamically allocate introspection.
We emphasize practicality: most heavy lifting remains hybrid, with classical optimizers training parameterized quantum circuits (PQCs) and classical monitoring safeguarding the device.

\section{Background and Design Principles}

\subsection{Transformers and KV Caches (Classical)}
Given token embeddings $x_t\in\mathbb{R}^d$, a Transformer forms queries $Q=XW_Q$, keys $K=XW_K$, and values $V=XW_V$.
Attention weights are $A=\softmax\big(\tfrac{QK^\top}{\sqrt{d}}\big)$ and outputs are $AV$.
The \emph{KV cache} stores $(K,V)$ from previous steps, allowing new queries to attend to past context at $O(Ld)$ rather than $O(L^2 d)$ cost.

\subsection{Quantum Primitives Used Here}
We rely on: (a) overlap estimation via the SWAP/Hadamard test to approximate $\abs{\qip{q}{k}}^2$; (b) amplitude estimation for faster mean/inner-product estimation; (c) block-encodings and polynomial approximations (QSVT) to emulate smooth functions on spectra; (d) \emph{QRAM} to index quantum states or circuits; (e) QEC to stabilize qubits.

\subsection{Constraints}
Two constraints shape our design:
(i) \textbf{No-cloning}.
Unknown quantum states cannot be freely copied; a naive cache of many identical quantum values is invalid.
(ii) \textbf{Reversibility}.
Transformers are not unitary by default (softmax and residuals are dissipative).
Quantum modules must be implemented as isometries/unitaries with either reversible embeddings or measurement-driven post-processing.

\section{Quantum-Native KV Cache (qKV)}

\subsection{Two qKV Realizations}
We define the qKV cache as a content-addressable memory supporting retrieval conditioned on a \emph{query state} $\qket{q}$.

\paragraph{(A) State Cache.}
Store $\{(\qket{k_j},\qket{v_j})\}_{j=1}^M$ under QEC in a memory register with QRAM-style addressing.
Retrieval proceeds by estimating overlaps $s_j = \abs{\qip{q}{k_j}}^2$ and preparing a distribution over indices $j$ proportional to $s_j$ (or a temperature-smoothed variant).
Because cloning is prohibited, reuse across attention heads can be implemented via \emph{coherently controlled access} and uncomputation rather than replication.
When multiple copies are required, values must be \emph{re-prepared} (see below) or accessed sequentially with uncompute.

\paragraph{(B) Program Cache.}
Instead of storing $\qket{v_j}$, store a classical description of a \emph{value-generating unitary} $U_{V_j}$ and (optionally) of $U_{K_j}$.
Upon retrieval, apply $U_{V_j}$ to a fresh workspace to \emph{regenerate} $\qket{v_j}$ on demand.
This circumvents no-cloning while keeping the cache durable and composable.
In practice, $U_{V_j}$ can be a PQC specified by parameters learned during training.

\subsection{Weighting and Retrieval}
Let a query register hold $\qket{q}$, and memory provide coherent access to $\qket{k_j}$ or $U_{K_j}$.
Using $R$ rounds of SWAP/Hadamard tests (or amplitude estimation) we obtain estimators $\hat{s}_j\approx \abs{\qip{q}{k_j}}^2$.
Define
\begin{equation}
    \alpha_j \propto \exp\!\Big(\beta\, \hat{s}_j\Big), \qquad 
    \sum_j \alpha_j = 1,
\end{equation}
as a softmax-like weight (temperature $\beta$), implemented via a polynomial approximation and linear combination of unitaries (LCU) within a block-encoding, or approximated by rejection sampling with amplitude amplification.

In \emph{state cache} mode, we prepare
\begin{equation}
    \sum_{j=1}^M \sqrt{\alpha_j}\,\qket{j}\qket{v_j},
\end{equation}
by first forming $\sum_j \sqrt{\alpha_j}\,\qket{j}$ and then conditionally loading $\qket{v_j}$ via QRAM/QROM or regenerating it with $U_{V_j}$.
The attended value is either measured or kept coherent to feed the next unitary block.

\begin{algorithm}[t]
\caption{qKV Retrieval with Overlap-Weighted Sampling}
\label{alg:qkv}
\begin{algorithmic}[1]
\Require Query $\qket{q}$, memory access to $\qket{k_j}$ or $U_{K_j}$, and value generators $U_{V_j}$
\State For each $j\in\{1,\dots,M\}$ estimate $\hat{s}_j \approx \abs{\qip{q}{k_j}}^2$ (SWAP/Hadamard tests or amplitude estimation)
\State Compute (coherently) weights $\alpha_j \propto \exp(\beta \hat{s}_j)$ and prepare $\sum_j \sqrt{\alpha_j}\,\qket{j}$
\State Controlled by $\qket{j}$, apply $U_{V_j}$ to a fresh value register to produce $\sum_j \sqrt{\alpha_j}\,\qket{j}\qket{v_j}$
\State (Optional) Post-select, normalize via amplitude amplification, or measure $\qket{j}$ to sample a concrete index
\State Return the value register as the attended output; uncompute ancillas to maintain reversibility
\end{algorithmic}
\end{algorithm}

\paragraph{Remarks.}
(i) The \emph{program cache} is the practical default because it scales with classical memory and allows arbitrary reuse via regeneration.
(ii) When full normalization is costly, approximate softmax by low-degree polynomials composed with block-encodings, or by sampling-with-rejection calibrated by a classical controller.

\section{Quantum Attention and Reasoning in Superposition}

\subsection{Quantum Attention as Overlap-Weighted Mixing}
Let $\{\qket{k_j}, \qket{v_j}\}$ be (programmatically) generated from past context.
Define the attention output as
\begin{equation}
    \qket{y} \;\propto\; \sum_{j=1}^{M} \exp\!\Big(\tfrac{\beta}{2}\hat{s}_j\Big)\,\qket{v_j},
    \qquad \hat{s}_j \approx \abs{\qip{q}{k_j}}^2.
\end{equation}
In a fully coherent variant one maintains
\(
\sum_j \sqrt{\alpha_j}\,\qket{j}\qket{v_j},
\)
feeding $\qket{y}$ to subsequent unitary blocks.
A measurement-based variant samples $j$ and executes a \emph{stochastic} but unbiased forward pass; repeated runs approximate the classical expectation.

\subsection{Superposed Deliberation}
Classical reasoning often explores multiple candidate thoughts.
Quantum hardware can \emph{coherently} maintain a superposition of partial hypotheses:
\begin{equation}
    \qket{\Psi_{\text{reason}}} 
    \;=\; \sum_{h=1}^{H} \gamma_h \,\qket{\text{hyp}_h}\qket{\text{trace}_h},
\end{equation}
where $\qket{\text{trace}_h}$ stores latent evidence.
The model may (i) apply \emph{phase kickback} from a consistency oracle to reweight $\gamma_h$, (ii) perform amplitude amplification toward more self-consistent hypotheses, and (iii) project only when a confidence threshold is met (Sec.~\ref{sec:meta}).

\subsection{Training}
Parameters live in PQCs $U_\theta$ embedded in attention/key/value generators and consistency oracles.
Gradients can be estimated via the parameter-shift rule or stochastic finite differences, with classical optimizers (e.g., Adam) updating $\theta$.
Hybrid loss terms combine token log-likelihood with metacognitive penalties (calibration, inconsistency) and stability regularizers (gate-time budgets).

\section{Quantum Metacognition}
\label{sec:meta}

\subsection{Objectives}
Metacognition aims to: (a) estimate confidence; (b) detect inconsistency; (c) allocate extra introspective compute when needed.
We propose a two-level controller:

\paragraph{Level 1: Fast Observables.}
Use \emph{classical shadows}/randomized measurements to estimate low-dimensional summaries (norms, pairwise overlaps, entanglement proxies) of the current belief superposition without full tomography.
Let $C$ denote a calibrated confidence observable; we target a mapping $C \mapsto \hat{p}_{\text{correct}}$ used for stopping or branching decisions.

\paragraph{Level 2: Deep Introspection.}
When $C$ is low or contradictions are detected, the controller triggers (i) additional rounds of overlap/consistency checks, (ii) selective uncomputation and re-branching, or (iii) a fall-back to more classical inference for stability.

\subsection{Consistency via Phase Kickback}
Define a \emph{consistency oracle} $\Upsilon$ acting as
\begin{equation}
    \Upsilon: \qket{\text{hyp}_h}\qket{0} \mapsto 
    \qket{\text{hyp}_h}\Big(\sqrt{1-\epsilon_h}\,\qket{0} + \sqrt{\epsilon_h}\,\qket{1}\Big),
\end{equation}
where $\epsilon_h$ encodes contradictions (computed by checking local constraints learned during training).
Amplitude amplification conditioned on ancilla $\qket{1}$ suppresses inconsistent hypotheses.

\subsection{Calibration and Self-Consistency}
We calibrate confidence by minimizing a Brier score between measured acceptance frequencies of $\Upsilon$ and a target probability.
A \emph{self-consistency prior} encourages final samples to agree across independently prepared runs; disagreement elevates $C$'s request for deeper introspection or external tools.

\section{Classical AI for Stabilizing Quantum Hardware}

Quantum AI is only useful if the device remains within calibrated tolerances.
We deploy classical AI in a high-rate control loop:

\paragraph{(i) Learned Decoders for QEC.}
Neural decoders (graph neural networks, transformers on syndrome graphs) provide fast approximate decoding for surface/LDPC codes, trading a small optimality gap for latency suitable for real-time control.

\paragraph{(ii) Online Calibration and Drift Tracking.}
Bayesian optimization and reinforcement learning (RL) tune pulse amplitudes, frequencies, and timings.
A \emph{digital twin} predicts gate infidelities under environmental drift; a classical controller schedules recalibrations or adjusts compilation on the fly.

\paragraph{(iii) Pulse-Level Optimal Control.}
Deep learning surrogates approximate GRAPE-type optimizers, proposing near-time-optimal pulses subject to robustness constraints (filter functions, bandwidth, leakage).

\paragraph{(iv) Resource Arbitration.}
A scheduler, informed by metacognitive signals ($C$), decides when to spend extra cycles on amplitude estimation or deeper introspection vs.\ advancing decoding, balancing latency and quality.

\begin{algorithm}[t]
\caption{AURORA: AI-Upregulated Robust Orchestrated Runtime for QAI}
\label{alg:aurora}
\begin{algorithmic}[1]
\Require Quantum model $U_\theta$, QEC stack, drift monitors, metacognition threshold $\tau$
\While{requests stream in}
  \State Run QEC; decode syndromes via neural decoder; if residuals $>$ budget, trigger recalibration
  \State Execute one decoding step of QAI; compute fast observables (classical shadows) to estimate confidence $C$
  \If{$C < \tau$}
     \State Allocate more introspection: extra overlap tests, deeper amplitude estimation, or re-branching
     \State RL agent updates pulse setpoints to keep error rates within guardrails
  \Else
     \State Continue with standard depth/latency budget
  \EndIf
  \State Log telemetry; update digital twin; schedule background calibrations
\EndWhile
\end{algorithmic}
\end{algorithm}

\section{Putting It Together: A Quantum Transformer}

Table~\ref{tab:components} summarizes the stack.
A typical layer takes a query register $\qket{q}$, uses the qKV cache to mix values, passes through a PQC nonlinearity (an isometry on an expanded register), and hands off to the next layer.
A small classical head monitors observables and steers control.

\begin{table}[h]
\centering
\caption{Quantum Transformer components and their realizations.}
\label{tab:components}
\begin{tabular}{@{}lll@{}}
\toprule
\textbf{Component} & \textbf{Quantum Realization} & \textbf{Notes} \\
\midrule
Embedding & State preparation $x\mapsto \qket{x}$ or PQC encoder & Basis/amplitude/angle encodings \\
Keys/Values & Program cache $\{U_{K_j},U_{V_j}\}$ & Regenerate on demand; avoids no-cloning \\
Similarity & SWAP/Hadamard tests; amplitude estimation & Approximates $\abs{\qip{q}{k_j}}^2$ \\
Weighting & LCU/block-encoding or sampling & Softmax-like normalization \\
Mixing & Controlled QROM/QRAM or QROM+PQC & Prepares $\sum_j \sqrt{\alpha_j}\,\qket{v_j}$ \\
Nonlinearity & PQC isometries + ancillas & Reversible residuals via widen-and-uncompute \\
Metacognition & Classical shadows; oracles $\Upsilon$ & Confidence \& consistency checks \\
Stabilization & Neural QEC decoders; RL calibration & Keeps device within guardrails \\
\bottomrule
\end{tabular}
\end{table}

\section{Resource Estimates (Back-of-the-Envelope)}

Let $d$ be embedding width, $M$ the cache size, $R$ the rounds of overlap estimation, and $H$ the number of reasoning hypotheses kept coherent.

\paragraph{Qubits.}
Working registers: $O(d)$ for embeddings, $O(\log M)$ for indices, $O(a)$ ancillas for tests and block-encodings, plus QEC overhead (code distance $D$) multiplying physical qubits by $\Theta(D^2)$ for surface codes.
Program cache adds classical memory; state cache adds $O(Md)$ logical qubits.

\paragraph{Depth.}
Per attention, depth $\approx R\cdot \mathrm{depth}(\text{overlap}) + \mathrm{depth}(\text{weighting}) + \mathrm{depth}(\text{mixing}) + \mathrm{depth}(U_\theta)$.
Amplitude estimation can reduce $R$ for target precision $\epsilon$ from $O(1/\epsilon^2)$ to $O(1/\epsilon)$ up to constants.

\paragraph{Latency vs.\ Quality.}
Metacognition increases latency when $C<\tau$, but prevents wasteful rollouts or low-confidence commits.
The scheduler can cap introspection rounds to meet SLOs.

\section{Evaluation Plan}

\paragraph{Phase I: Sim2Real on Noise Models.}
Benchmark overlap estimation accuracy, weighting stability, and PQC training in noisy simulators with realistic QEC latencies.
Report calibration curves for confidence $C$ (Brier score, ECE).

\paragraph{Phase II: Toy Reasoning Tasks.}
Arithmetic chains, symbolic puzzles, and structured retrieval (few-shot) where content addressing and superposed hypotheses are beneficial.
Metrics: accuracy vs.\ latency; self-consistency rate; abstention quality; energy budget.

\paragraph{Phase III: Hybrid Language Tasks.}
On small-vocabulary synthetic corpora, compare (i) purely classical transformer baselines, (ii) quantum-assisted attention with program cache, and (iii) full QAI with metacognition.
Ablations quantify the contribution of qKV and metacognition independently.

\section{Limitations and Open Problems}

\textbf{QRAM practicality.} Bucket-brigade QRAM remains an engineering challenge.
Our default to \emph{program} caching is a pragmatic compromise.

\textbf{No-cloning tension.} True reuse of unknown values across many heads is not possible; regeneration or sequential access with uncompute is required.

\textbf{Normalization cost.} Softmax-like weighting in a unitary framework is nontrivial.
Polynomial/LCU approximations or stochastic sampling introduce bias/variance trade-offs.

\textbf{Training stability.} Barren plateaus and gradient noise can hinder PQC training.
Curricula, layerwise pretraining, and classical warm starts may be necessary.

\textbf{Hardware scale.} QEC overheads dominate; near-term instantiations will be small and problem-specific.
Advantages, if any, may appear first as \emph{quality-of-reasoning} improvements under fixed latency rather than raw speedups.

\section{Related Concepts (Pointers)}
We align with and extend threads from quantum machine learning (QML), quantum associative memory, QRAM-based data structures, amplitude estimation, classical shadows for measurement-efficient property estimation, neural decoders for QEC, and ML-driven calibration/optimal control.
Our contribution is a \emph{system-level} design that ties these ingredients to a Transformer-like generative model with an explicit metacognitive controller and a classical-AI stabilization loop.

\section{Conclusion}

We outlined a blueprint for a quantum-native generative model.
By making the KV cache and attention quantum (with program-centric caching to respect no-cloning), enabling superposed reasoning with metacognitive gating, and wrapping the device in a classical-AI control stack for stabilization, the proposal offers a path to explore \emph{qualitative} benefits of quantum computation for reasoning, long-context memory, and introspection.
The next steps are empirical: build minimal working prototypes, quantify calibration and consistency benefits, and measure end-to-end quality/latency trade-offs under realistic QEC and control constraints.

\paragraph{Acknowledgments}
We thank colleagues in quantum control, error correction, and classical LLMs for discussions that shaped this design.

\appendix

\section{Hadamard and SWAP Tests for Similarity}

To estimate $\Re\,\qip{q}{k}$, prepare $\qket{0}\qket{q}$, apply a controlled-$U$ with $U\qket{k}=\qket{q}$ (or prepare both states and use the Hadamard test), then measure the ancilla in $X$ basis; repeat to concentrate the estimate.
For $\abs{\qip{q}{k}}^2$, the SWAP test measures the overlap by controlling a SWAP between $\qket{q}$ and $\qket{k}$ with an ancilla initialized in $\qket{+}$.

\section{Reversible Residuals}
To maintain reversibility, widen the channel with ancillas, apply an isometry approximating the residual nonlinearity, and then uncompute ancillary workspaces after the next block commits its state.

\section{Pseudocode: Metacognitive Gate}
\begin{algorithm}[h]
\caption{Metacognitive Gating with Confidence Observable}
\label{alg:meta}
\begin{algorithmic}[1]
\Require Confidence threshold $\tau$, budget $B$
\State Compute fast confidence $C$ via classical shadows of current registers
\If{$C < \tau$ and $B>0$}
  \State Run $k$ additional overlap/consistency checks; update $C$
  \State $B \gets B - k$
  \If{$C$ still $< \tau$} invoke fallback: more classical inference or abstain
\EndIf
\State Commit measurement; proceed to next token/layer
\end{algorithmic}
\end{algorithm}

\section*{Selected References (Non-exhaustive)}
\vspace{-0.5em}
\begin{itemize}\itemsep0.2em
\item S.\ Lloyd, M.\ Mohseni, P.\ Rebentrost, ``Quantum algorithms for supervised and unsupervised machine learning,'' \emph{arXiv:1307.0411}.
\item J.\ Biamonte, et al., ``Quantum machine learning,'' \emph{Nature} 549, 195--202 (2017).
\item N.\ Wiebe, D.\ Braun, S.\ Lloyd, ``Quantum algorithm for data fitting,'' \emph{Phys.\ Rev.\ Lett.} 109, 050505 (2012).
\item A.\ W.\ Harrow, A.\ Hassidim, S.\ Lloyd, ``Quantum algorithm for linear systems of equations,'' \emph{Phys.\ Rev.\ Lett.} 103, 150502 (2009).
\item V.\ Giovannetti, S.\ Lloyd, L.\ Maccone, ``Quantum random access memory,'' \emph{Phys.\ Rev.\ Lett.} 100, 160501 (2008).
\item S.\ Aaronson, ``Shadow tomography of quantum states,'' \emph{STOC} (2018).
\item H.-Y.\ Huang, R.\ Kueng, J.\ Preskill, ``Predicting many properties of a quantum system from very few measurements,'' \emph{Nat.\ Phys.} 16, 1050–1057 (2020).
\item L.\ K.\ Grover, ``A fast quantum mechanical algorithm for database search,'' \emph{STOC} (1996) / \emph{arXiv:quant-ph/9605043}.
\item A.\ Fowler, M.\ Mariantoni, J.\ Martinis, A.\ Cleland, ``Surface codes: Towards practical large-scale quantum computation,'' \emph{Phys.\ Rev.\ A} 86, 032324 (2012).
\item I.\ Cong, S.\ Choi, M.\ Lukin, ``Quantum convolutional neural networks,'' \emph{Nat.\ Phys.} 15, 1273–1278 (2019).
\item P.\ K.\ F\"osel, et al., ``Reinforcement learning with neural networks for quantum feedback,'' \emph{Phys.\ Rev.\ X} 8, 031084 (2018).
\item N.\ P.\ O'Neill, C.\ Chamberland, ``Neural decoders for quantum error-correcting codes,'' survey/preprints.
\end{itemize}

\end{document}
